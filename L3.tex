\documentclass{./template}
  \usepackage{stmaryrd}

\title{Commutative Algebra and Algebraic Geometry --- Lecture 3}
\author{Martin Svanberg}

\newcommand{\Lead}{\textrm{Lead }}
\newcommand{\Tor}{\textrm{Tor}}
\newcommand{\Ker}{\textrm{Ker}}
\newcommand{\Img}{\textrm{Im}}
\newcommand{\Hom}{\textrm{Hom}}
\newcommand{\coker}{\textrm{coker}}

\usetikzlibrary{decorations.pathmorphing}


\newcommand\xrsquigarrow[1]{%
    \mathrel{%
        \begin{tikzpicture}[%
            baseline={(current bounding box.south)}
            ]
        \node[%
            ,inner sep=.44ex
            ,align=center
            ] (tmp) {$\scriptstyle #1$};
        \path[%
            ,draw,<-
            ,decorate,decoration={%
                ,zigzag
                ,amplitude=0.7pt
                ,segment length=1.2mm,pre length=3.5pt
                }
            ] 
        (tmp.south east) -- (tmp.south west);
        \end{tikzpicture}
        }
    }

\begin{document}
  \maketitle

  \section{Modules cont'd}

  $R$ is commutative with 1 unless specified otherwise.

  {\rec Let $M$ on $R$-module $R\times M\to M$. 
  }

  There is no concept analogous to normal subgroups. You are allowed to take quotient at any time, due to commutatitvity.

  {\defn A function $f:M\to N$ of $R$-modules is a homomorphism of $R$-modules (or $R$-linear map)  if we have
  
  1. $f(\underbrace{rm}_\text{M action})=\underbrace{rf(m)}_\text{N action}$
  
  2. It's a homomorphism of the underlying Abelian group. That is, $f(m+n)=f(m)+f(n)$.
  
  }

  Let $X\subset M$. Then $X$ {\em generates} $M$ if $\forall m\in M \exists$ a finite number $r_1,\ldots,r_n\in R$ and $x_1,\ldots,x_n\in X$ st element $m=\sum r_ix_i$. Has to be finite because infinite sums not generally defined in rings, i.e. no general notion of convergence.
  
  If $X=\{x_1,\ldots,x_n\}$ write $M=(x_1,\ldots,x_n)=Rx_1+\ldots+Rx_n$.

  $M$ is {\em finitely generated} if $M$ is generated by $X$ for some finite $X\subset M$.
  
  $M$ is {\em cyclic} if $M=(x)$ for some $x\in M$. This is the analog of principal ideals. 

  \hrulefill

  $X\subset M$ is a basis (or free set of generators) if for all $m\in M$ there exists a unique decomposition $m=\sum r_ix_i$ where $r_i\in R\quad x_i\in X, \quad r_i,x_i\neq 0$.

  $M$ is {\em free} if there exists a basis for $M$.

  {\ex $R=k=\text{field}\implies M$ is vector space $\implies M$ free.}

  {\ex $R=\Z$ and $M$ finitely generated, then $M \text{free} \iff M \cong \Z^k$}

  {\ex $R=\Z$. $\Z/n\Z$ not free since $n\cdot 1 = 0$ but cyclic generated by $1\in \Z/n\Z$.}

  {\defn Torsion. $M\in R$-module. $\Tor(M)=\{ m\in M \mid \exists r\in R $ st $ rm=0, r\neq 0\}$
  }

  {\exe $R$ domain $\implies \Tor(M)$ is a submodule.  }

  $M$ is {\em torsion-free} if $\Tor(M)=(0)$.

  $M$ is torsion if $M=\Tor(M)$.

  {\ex $k=\R=$ field. There is no torsion in a vector space.}

  If $R=\Z$, $M$ Abelian group $\implies \Tor(M) = \{m\in M \mid $  order of $M$ finite $\}$.

  $km$ additively $\leftrightarrow$ $m^k$ multiplicatively

  $km=0 \leftrightarrow$ $m^k=1$

  $\Tor(\Z^k)=(0)$. Every finite Abelian group is a torsion $\Z$-module.
  
  Torsion and freeness are opposite extremes.

  Let $X$ any set. $R^X=\{ $collection of {\bf finite} sums $\sum r_ix_i \}=\bigoplus_{x\in X} Rx$ (or $F(X)$ in Dummit-Foote). This is the free module generated by $X$.

  Universal property: inclusion $\imath : X\to R^X, x\mapsto x$. For all $M \in R$-module, for all set maps $f:X\to M$ there exists a unique extension $F:R^X\to M$ st $F(x)>f(x)$.

  \begin{center}
\begin{tikzcd}
  X \arrow[r, "\imath"] \arrow[rd, "f"'] & R^X \arrow[d, "\exists !"] \\
   & M
  \end{tikzcd}
\end{center}

  $X$ is just a set. The unique map is a module homomorphism.

  Remark. For groups, Free$(x)$= free group on $X$. $F_n$: see chap 6 of DF. Nonabelain for $n\geq 2$.

  \hrulefill

  Observe. Every module is quotient of a free module. E.g. Take $X=M$ as a set. $f=$ Id.

  $F:R^X\to M$ is surjective $\iff$ $f(X)$ generates $M$.

  \hrulefill

  Assume $R^X \xrightarrow F M$. Then $\Ker F = $ module of relations. $M$ is finitely presented if the re exists a generated set $X$ st 1) $X$ is finite 2) $\Ker F$ is finitely generated.

  $(*)$ If $R$ Noetherian and $M$ is finitely generated  $\implies$ all submodules of $M$ are finitely generated. So $R$ Noetherian implies every finitely generated module is finitely presented.

  .e.g free groups $\rightsquigarrow$ presentation of groups. $D_8=\langle r,s\mid r^4=s^2=e, srs^{-1}=r^3=r^{-1}\rangle$

  $F_2\subseteq $ Free $(r,s)$ $\twoheadrightarrow D_8$ (arrow means surjective?)

  $\Ker=($ normal subgroup generated by $r^4$, $s^2$, $srs^{-1})$

  \hrulefill

  ACC for $M\in R$-module means $\forall M_1\subset M_2\subset\ldots\subset M_k\subset M$ (ascending chain of $R$ modules $\subset M$) there exists $t$ st $M_t=M_{t+1}=\ldots$

  {\defn $M$ is Noetherian if $M$ satisfies ACC. $R$ is Noetherian as a ring iff $R$ is Noetherian $R$-module, all submodules of $R$ are ideals of $R$. Lemma about Noetherian rings works for $R$-modules. That is,}
  {\lemma TFAE:
  \begin{enumerate}[(i)]
   \item{$M$ satisfies ACC}
   \item{Every nonempty collection of submodules has a maximal element}
   \item{Every submodule is finitely generated} 
  \end{enumerate}
  {\proof Same as earlier.}
  }

  Lemma 1 implies $(*)$.

  {\thm Classification of finitely generated modules over a PID. Let $R$ be a PID and $M$ an $R$-module, finitely generated. Then $\exists k\in \Z$ st
  $$M \cong \R^k \oplus \Tor(M)$$
  A PID is by definition an integral domain, so this is a submodule.
  $$\Tor(M)\cong \underbrace{R / (a_1)\oplus\ldots\oplus R/(a_k)}_\text{each cyclic torsion generated by 1}$$
  with nonunits $a_i\in R$, $a_1 \mid a_2 \mid \ldots \mid a_k$.

  Remark: $R^k\cong R^X$ with $\abs{X}=k$. $R^k$ is like $\underbrace{R\oplus\ldots\oplus R}_\text{k times}$ each $R$ is cyclic.

  In particular: every finitely generated $R$-module is a direct sum of cyclic modules. Importantly, $M$ is torsion free $\iff$ $M$ is free.
  }
  
  {\ex If $R=\Z$: fundamental theorem of finitely generated Abelian groups. }

  $M$ an $R$-module $\Leftrightarrow$ $(V,T)$ $V$ is vector space over $k$ and $T : V\to V$.
  
  $\dim_k V < \infty$ $\iff$ free part $k[x]^n=0$.
  
  Jordan normal form of $T \leftrightarrow$ $M\cong k[x]/a_1(x)\oplus\ldots\oplus k[x]/a_n(x)$

  The proof of Thm 1 is more boring than hard.

  \section{Exact sequences of modules}

  Given a homomorphism of modules, $\to M_{i-1}\xrightarrow f M_i \xrightarrow g M_{i+1}$ is exact at $M_i$ if $\Ker g = \Img f$.

  e.g. $0 \to M' \xrightarrow f M$ exact $\iff$ $f$ injective
  
  $M \xrightarrow g M'' \to 0$ exact $\iff$ $g$ surjective
  
  $\underbrace{0 \xrightarrow f M'\xrightarrow g M \to M'' \to 0}_\text{(short exact sequence)}$ is exact $\iff$ $M/\Img f \cong M''\quad$ ($\Img f \cong M'$)

  Given $M \xrightarrow f M''$ a {\em section} of $f$ is a map $s:M''\to M$ st $f\circ s = Id_{M''}$. 

  {\ex A vector field is a section of a tangent bundle.}

  $0\to M' \xrightarrow f M \xrightarrow g M'' \to 0$ is {\em split} if there exists section $s:M''\to M$ of g. Then $s$ is called {\em splitting} and we get $M\cong M' \oplus M''$.

  In general, say $M$ is an extension of $M''$ by $M'$.

  Analogy with arbitrary groups: $G$ group, $N$ normal in $G$. $H=G/N$. Having this data is the same as having $1 \to N \to G \to \underbrace{G/N}_{=H} \to 1$ "multiplicatively written short exact sequence"

  Again, split if $\exists$ group homomorphism $s: H\to G$.

  Observe: Split for groups does not in general imply $G\cong N\times G/N$. Only for Abelian groups. If split, then we have $G=N\cdot s(H),\quad N\neg s(H)=\{1\} \iff G=N \rtimes H$ (semidirect product)

  $f:M\to N$ have of $R$-modules $\coker(f)=N/\Img f$.

  Observe: For groups cokernel may not exist.

  \hrulefill

  {\color{red} This part was transcribed in haste. Read the book instead.}

  $\Hom(M,N)=\{ R$-linear maps $f:M\to N\}$, an Abelian group.

  \begin{center}
  \begin{tikzcd}
    & L \arrow[d] \arrow[rd] &  &  &  \\
   0 \arrow[r] & M' \arrow[r] & M \arrow[r] \arrow[rd] & M'' \arrow[r] \arrow[d] & 0 \\
    &  &  & L' & 
   \end{tikzcd}
  \end{center}

  
  \begin{center}
  \begin{tikzcd}
    M' \arrow[r, "f"] & M &  \xrsquigarrow{~~~~~~~~~~~} & {\Hom(L,M)} \arrow[r, "f_*"] & {\Hom(L,M')}
    \end{tikzcd}
  ($\alpha : L\to M'$, and $f_*(\alpha)=f\circ \alpha$)
  \end{center}

  \begin{center}
  \begin{tikzcd}
    M \arrow[r, "g"] & M'' \arrow[r] & {\Hom(M'',L')} \arrow[r, "g^*"] & {\Hom(M,L)} & 
    \end{tikzcd}
  ($\beta:M''\to L', g^*(\beta)=\beta\circ g$)
  \end{center}


  $(*)\quad 0\to M' \to M\to M'' \to 0$ exact $\implies$ $0\to \Hom(L,M')\to\Hom(L,M)\xrightarrow {(\dagger)}\Hom(L,M'')$ exact but maybe not $"\rightarrow 0"$. $(\dagger)$ maybe not surjective

  {\defn $L$ is projective $\iff \forall 0\to M'\to M\to M''\to 0$, $\Hom(L,M)\to \Hom(L,M'')$ surjective.}

  Dually, apply $\Hom(\_,L)$ to $(*)$ get
  $$\Hom(M',L)\leftarrow \Hom(M,L) \leftarrow \Hom(M'',L)\leftarrow 0$$
  exact but maybe not $"0\leftarrow"$

  {\defn $L$ is injective $\iff$ $\forall (*)$ $\Hom(\_,L)(*)$ again short exact sequence}

  Thoughts:
  \begin{itemize}
    \item{Free $\implies$ projective}
    \item{Projective $\iff$ direct sum of free}
    \item{$R$ PID $\implies$ $M$ finitely generated ($M$ \text{projective} $\iff$ $M$ free)}
  \end{itemize}

  $R=k[x_1,\ldots,x_n]$ $n>2$ not PID. Serre's question: every finitely generated projective module over a polynomial ring is free. Quillen–Suslin theorem proves Serre's problem.

\end{document}
