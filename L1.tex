\documentclass{./template}

\title{Commutative Algebra and Algebraic Geometry --- Lecture 1}
\author{Martin Svanberg}

\begin{document}
  \maketitle
  \section{Books}
  \begin{itemize}
  \item{Main book: Dummit and Foote ch. 15.}
  \item{Görtz-Wedhorn ch. 1.}
  \item{Atiyah-Macdonald}
  \end{itemize}

  \section{Musings on the nature of geometry}
  Manifolds look locally like $\mathbb R^n$.

  We can consider geometric objects as sets of solutions to polynomial equations in $\mathbb A_k^n$. Affine space is especially useful in algebraic geometry.

  $\mathbb A_k^n=$affine space of dimension n/k$=\{(x_1,\ldots,x_k)\in k^n\}$.
  $$A_{\mathbb R}^n=\mathbb R^n$$
  In $\mathbb R$ or $\mathbb C$ you get manifolds from this if the solutions are smooth.

  Complex manifolds look locally like $\mathbb C^n$. Key distinction is holomorphism, which is a far stricter requirement than just (real) differentiability.

  {\thm Riemann. Assume $X$ is a compact Riemann surface (locally looks like $\mathbb C$). Then a curve in it is algebraic, i.e. it is the complex solution of a system of polynomial equations in some $\mathbb P^n(\mathbb C)$.}

  e.g. a compact Riemann surface (torus) $\mathbb C/(\mathbb Z + i\mathbb Z)  \quad c\to y_2^2=x^3-xz^2$.

  Projective space is lines through the origin in $k^{n+1}$:
  $$\mathbb P_k^n = (k^{n+1}\setminus 0) / \sim$$
  where $(a_0,\ldots,a_{n+1}) \sim (\lambda a_0,\ldots, \lambda a_{n+1})$.

  $\mathbb P_k^n(\mathbb C)$ is compact.

  We study local behavior in this course. One of the nontrivial insights of Grothendieck is having a natural notion of functions on prime ideals.

  \section{Noetherian Rings}
  The most basic type of ring in commutative algebra. Non-notherian rings are wild. For most of this course, let $R$ be a commutative ring with 1. We assume $R$ is a commutative ring.

  \rec A ring is abelian group under addition, group under multiplication (without zero), not generally assumed to be commutative

  {\rec $R$ is an integral domain if it has no non-zero zero divisors. This allows cancellation, i.e. $ab=ac \implies b=c$.}

  {\defn $R$ is Noetherian if $R$ satisfies the Ascending Chain Condition on ideals (ACC). That is, if you have an ascending chain of ideals $I_1 \subset I_2 \subset \ldots$ it becomes stationary --- there exists an $n_0$ such that $I_n=I_{n_0}$ for all $n\geq n_0$.}

  Noetherian does not require or imply integral domain.

  {\defn $R$ is Artinian if $R$ satisfies the Descending Chain Condition on ideals. That is, $J_1 \supset J_2 \supset \ldots$ becomes stationary.}

  {\ex A finite ring is both Noetherian and Artinian.}

  {\rec A principal ideal domain is a domain where every ideal can be generated by a single element. $R$ is a PID if $R$ is an integral domain and every ideal in $R$ is principal.}

  {\ex $\mathbb Z$ is a principal ideal domain.}

  $$(n) \subset (m) \iff m \mid n.$$ An ascending chain $\ldots \mid n_3 \mid n_2 \mid n_1$ in $Z$ gives divisors of $n_1$. $n_1$ has finitely many divisors $\implies \mathbb Z$ is Noetherian.

  The multiples of two contain the multiples of four which contain the multiples of eight, and so on. This is an infinite descending chain and is thus not Artininan.

  {\ex Fields are Noetherian and Artinian since they only have one proper (that is not the whole thing) ideal.}
  
  For fields, $0\neq 1$ and all elements have inverses.

  {\ex $R\times R\times \ldots \times R = R^{\mathbb N}$. $I_1=(R,0,\ldots,0) \subset I_2=(R,R,0,\ldots)\subset \ldots$. Not Noetherian.}

  {\ex $k[x_1,\ldots,x_n]$ in infinitely many variables. $(x_n)\subsetneq(x_1,x_2)\subsetneq(x_1,x_2,\ldots)$ not Noetherian. But $k[x_1,\ldots,x_n]$ is a unique factorization domain.}

  Fields $\subset$ Euclidean domains $\subset$ PIDs $\subset$ UFDs $\subset$ integral domains.

  {\rec $u\in R$ is a unit if $\exists v\in R$ st $uv=1$. $R^\times$ is the group of units.}
  {\defn $a\in R$ is irreducible if $a\not\in R^\times$ and whenever $a=bc$, either $b\in R^\times$, $c\in R^\times$.}

  {\defn If $a,b\in R^\times$ and $a,b\neq 0$ we say $a,b$ are associate if $\exists u\in R^\times$ st $a=ub$, e.g. 5 and -5 are associate in $\mathbb Z$.}

  {\defn $R$ is a UFD if $R$ is a domain and every nonzero nonunit $a\in R$ has a unique factorization.

  \begin{enumerate}[1)]
  \item{ Exists:can write there is a factorization $a=p_1^{e_1}\ldots p_r^{e_r}$ such that $p_i$ irreducible for all $i$ and $p_i,p_j$ are not associate for distinct $i,j$}

  \item{If there is a second factorization $q_1^{e_1}\ldots q_s^{e_s}$ then $r=s$, and after reordering the factors $q_i$ is associate to $p_i$ etc.}
  \end{enumerate}
  }

  {\ex $k[x_1,\ldots,x_n]$ is Noetherian. To be covered next time.}

  {\thm Hilbert Basis Theorem: if $R$ Noetherian then $R[x]$ is also Noetherian.}

  {\lemma Quotients of Noetherian rings are Noetherian: $k[x_1,\ldots,x_k]/I$ is Noetherian for all $\forall k$, $\forall n$ for all ideals $I$.}

  $k[x_1,\ldots,x_k]/I$ is important. Sets of solutions to these polynomials are key to algebraic geometry.

  {\lemma $R$ is Noetherian $\iff$ every ideal is finitely generated.}

\end{document}
